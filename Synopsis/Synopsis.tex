\documentclass[a4paper,12pt,table]{article}
\usepackage[margin=1in]{geometry}
\usepackage{amsmath}
\usepackage{amssymb}
\usepackage{fancyhdr}
\usepackage[table,xcdraw]{xcolor}
\usepackage{float}
\usepackage{blindtext}
\usepackage{microtype}
\usepackage[hidelinks]{hyperref}
\usepackage{graphicx}
\usepackage[ruled,vlined]{algorithm2e}
\usepackage{color,soul}


\renewcommand{\arraystretch}{2.5}

\setlength{\parindent}{0cm}
\setlength{\parskip}{1em}
\pagestyle{fancy}
\fancyhf{}
\renewcommand{\headrulewidth}{0pt}
\renewcommand{\footrulewidth}{0pt}
\rhead{
}
\fancyfoot[R]{\thepage}
\graphicspath{ {C:/Users/sarya/Desktop/Semester 4/ISM/Report} }

\begin{document}
    
\begin{titlepage}
    \begin{center}
        \vspace*{6cm}
 
        \textbf{{\Huge Independent Study Module Synopsis}}
 
     \vspace{5cm}
 
  {\Large An Investigation of Reinforcement Learning Algorithms for Mastering the Game of Santorini}
 
        \vspace{0.5cm}
 
        \vspace{1.5cm}
 
 
        \vfill
 
 
 
        \textbf{Aryan Sarswat -  A0200521E \\
         Lim Wei Liang - A0205466E \\
         Roy Chua Dian Lun - A0199930N \\ }
 
        \vspace{0.8cm}
 
       University Scholars Programme\\
        National University of Singapore\\
 
    \end{center}
 \end{titlepage}

 \newpage

\section{Synopsis}

Reinforcement learning is a technique that has been successfully applied to various board games, such as Chess and Go. This paper will be exploring similar methods, such as value function approximation and Monte Carlo Tree Search, to optimize a similar board game, Santorini, and attempt to come up with an optimal way to play and win the game. \par



\end{document}