\documentclass[a4paper,12pt,table]{article}
\usepackage[margin=1in]{geometry}
\usepackage{amsmath}
\usepackage{amssymb}
\usepackage{fancyhdr}
\usepackage[table,xcdraw]{xcolor}
\usepackage{float}
\usepackage{blindtext}
\usepackage{microtype}
\usepackage[hidelinks]{hyperref}
\usepackage{graphicx}
\usepackage[ruled,vlined]{algorithm2e}
\usepackage{color,soul}


\renewcommand{\arraystretch}{2.5}

\setlength{\parindent}{0cm}
\setlength{\parskip}{1em}
\pagestyle{fancy}
\fancyhf{}
\renewcommand{\headrulewidth}{0pt}
\renewcommand{\footrulewidth}{0pt}
\rhead{
}
\fancyfoot[R]{\thepage}
\graphicspath{ {C:/Users/sarya/Desktop/Semester 4/ISM/Report} }

\begin{document}
    
\begin{titlepage}
    \begin{center} 
        \textbf{{\Huge Independent Study Module Synopsis}}
    \vspace{3cm}
    
    \end{center}
  \Large In the field of artificial intelligence, reinforcement learning techniques have performed extremely well when applied to complex games, such as Chess and Go. Therefore, this study investigates the use of reinforcement learning techniques for effectively playing the tactical board game Santorini. A wide variety of methods, such as linear value function approximation, non-linear Neural Network-based approximators, and Monte Carlo Tree Search were implemented, tested, and compared with one another. Ultimately, we managed to develop an agent that is capable of playing Santorini at a higher level than regular humans. \par
 
        \vspace{0.5cm}
 
        \vspace{1.5cm}
 
 
        \vfill
 
 
        \begin{center}
            \textbf{Aryan Sarswat -  A0200521E \\
            Lim Wei Liang - A0205466E \\
            Roy Chua Dian Lun - A0199930N \\ }
    
           \vspace{0.8cm}
    
          University Scholars Programme\\
           National University of Singapore\\
        \end{center}
 \end{titlepage}
\end{document}